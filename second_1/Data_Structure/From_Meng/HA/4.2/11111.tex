\documentclass[paper=a4, fontsize=11pt]{scrartcl} % A4 paper and 11pt font size
\usepackage[noend]{algpseudocode}
\usepackage{ctex}
\usepackage[T1]{fontenc} % Use 8-bit encoding that has 256 glyphs
\usepackage{fourier} % Use the Adobe Utopia font for the document - comment this line to return to the LaTeX default
\usepackage[english]{babel} % English language/hyphenation
\usepackage{amsmath,amsfonts,amsthm} % Math packages
\usepackage[english]{babel}
\usepackage[utf8]{inputenc}
\usepackage{indentfirst}
\usepackage{algorithm}
\usepackage{lipsum} % Used for inserting dummy 'Lorem ipsum' text into the template

\usepackage{sectsty} % Allows customizing section commands
\allsectionsfont{\centering \normalfont\scshape} % Make all sections centered, the default font and smaıll caps

\usepackage{fancyhdr} % Custom headers and footers
\pagestyle{fancyplain} % Makes all pages in the document conform to the custom headers and footers
\fancyhead{} % No page header - if you want one, create it in the same way as the footers below
\fancyfoot[L]{} % Empty left footer
\fancyfoot[C]{\thepage} % Empty center footer
\fancyfoot[R]{} % Page numbering for right footer
\renewcommand{\headrulewidth}{0pt} % Remove header underlines
\renewcommand{\footrulewidth}{0pt} % Remove footer underlines
\setlength{\headheight}{13.6pt} % Customize the height of the header

\numberwithin{equation}{section} % Number equations within sections (i.e. 1.1, 1.2, 2.1, 2.2 instead of 1, 2, 3, 4)
\numberwithin{figure}{section} % Number figures within sections (i.e. 1.1, 1.2, 2.1, 2.2 instead of 1, 2, 3, 4)
\numberwithin{table}{section} % Number tables within sections (i.e. 1.1, 1.2, 2.1, 2.2 instead of 1, 2, 3, 4)

\setlength\parindent{0.3pt} % Removes all indentation from paragraphs - comment this line for an assignment with lots of text

%----------------------------------------------------------------------------------------
%	TITLE SECTION
%----------------------------------------------------------------------------------------

\newcommand{\horrule}[1]{\rule{\linewidth}{#1}} % Create horizontal rule command with 1 argument of height
\title{
\normalfont \normalsize
\textsc{Shanghai Jiao Tong University} \\ [25pt] % Your university, school and/or department name(s)
\horrule{0.5pt} \\[0.4cm] % Thin top horizontal rule
\huge Huffman System \\ % The assignment title
\horrule{2pt} \\[0.5cm] % Thick bottom horizontal rule
}

\author{\\ \kaishu 吕艺\\ \normalsize 517021910745} % Your name

\date{\normalsize\today} % Today's date or a custom date

\begin{document}

\maketitle % Print the title
\kaishu
\section{需求分析}

1.本演示文件中操作的主要数据结构为哈夫曼树,数的叶子个数不限。首先需要在主餐单中输入命令'I'来进行哈夫曼树的初始化(这是之后所有操作的前提条件)
为增强程序的健壮性,用户如果想在哈夫曼树初始化前就进行其他操作
显示错误信息并被要求重新输入。
初始化操作的输入为字符个数$n$(输入结束后按回车),字符(不带空格输入,输入结束后按回车)和他们对应的权重(每个整数之间以空格分割,输入结束后按回车键)
之后可进行的操作有编码('E',根据已建的哈夫曼树对给定文件中的字符),译码('D',利用已建的哈夫曼树将给定文件中的二进制码还原为字符),打印代码文件('P',打印在屏幕上并保存到文件中)和打印哈夫曼树('T',打印在屏幕上并以凹入表的形式保存在文件中)。
\vspace{0.5cm}

2.演示文件以用户和计算机的对话方式进行,即在计算机终端上显示合适的提示信息之后,由用户在键盘上输入演示程序中规定的运算命令;命令结束完后,生成相应的文件或在屏幕上显示结果。
\vspace{0.5cm}
\newpage
3.程序执行的命令包括:
\begin{enumerate}
    \item I:初始化(Initialization)。 从终端读入字符集大小 n,以及 n 个字符和 n 个权值, 建立哈夫曼树,并将它存于文件 hfmTree 中。
    \item E:编码(Encoding)。利用已建好的哈夫曼树,对文件 plainFile 中的正文进行编 码,然后将结果存入文件 codeFile 中。
    \item D:译码(Decoding)。利用已建好的哈夫曼树,对文件 codeFile 中的代码进行译 码,然后将结果存入文件 textFile 中。
    \item P:打印代码文件(code Printing)。将文件 codeFile 显示在终端上,每行 50 个代 码。同时将此字符形式的编码文件写入文件 codePrint 中。
    \item T:打印哈夫曼树(Tree printing)。将哈夫曼树以直观的方式(树或凹入表形式) 显示在终端中,同时将此字符形式的哈夫曼树写入文件 treePrint 中。
\end{enumerate}

\vspace{0.3pt}

4.测试数据
\begin{enumerate}
    \item  已知某系统在通信联络中可能出现 8 种字符,其出现频率分别为 0.05,0.29,0.07, 0.08,0.14,0.23,0.03 和 0.11。
    \item 用下表给出的字符集和频度的实际统计数据建立哈夫曼树,并实现以下报文中的 编码和译码:”THIS PROGRAM IS MY FAVORITE”。
    \begin{table}[ht]
        \caption{测试数据}
        \centering
        \begin{tabular}{|c c c c c c c c c c c c c c c|}
            \hline
            char & space & A & B & C & D & E & F & G & H & I & J & K & L & M \\
            \hline
            Weight & 186 & 64 & 13 & 22 & 32 & 103 & 21 & 17 & 47 & 57 & 1 & 5 & 33 & 20\\
            \hline
            char & N & O & P & Q & R & S & T & U & V & W & X & Y & Z& & \\
            \hline
            Weight & 57 & 63 & 15 & 1 & 49 & 51 & 80 & 23 & 8 & 19 & 1 & 16 & 1\\
            \hline
        \end{tabular}
    \end{table}
\end{enumerate}











% \begin{algorithm}
% \caption{Euclid’s algorithm}\label{alg:euclid}
% \begin{algorithmic}[1]
% \Procedure{Euclid}{$a,b$}\Comment{The g.c.d. of a and b}
% \State $r\gets a\bmod b$
% \While{$r\not=0$}\Comment{We have the answer if r is 0}
% \State $a\gets b$
% \State $b\gets r$
% \State $r\gets a\bmod b$
% \EndWhile\label{euclidendwhile}
% \State \textbf{return} $b$\Comment{The gcd is b}
% \EndProcedure
% \end{algorithmic}
% \end{algorithm}


\end{document}




% {
%     "latex-workshop.view.pdf.viewer": "browser",
%     "latex-workshop.intellisense.package.enabled": true,
%     "latex-workshop.intellisense.surroundCommand.enabled": true,
%     "latex-workshop.intellisense.unimathsymbols.enabled": true,
%     "latex-workshop.latex.autoBuild.onTexChange.enabled": true,
%     "window.zoomLevel": 0
% }
